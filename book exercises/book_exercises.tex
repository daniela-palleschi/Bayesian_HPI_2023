% Options for packages loaded elsewhere
\PassOptionsToPackage{unicode}{hyperref}
\PassOptionsToPackage{hyphens}{url}
%
\documentclass[
]{article}
\usepackage{amsmath,amssymb}
\usepackage{lmodern}
\usepackage{iftex}
\ifPDFTeX
  \usepackage[T1]{fontenc}
  \usepackage[utf8]{inputenc}
  \usepackage{textcomp} % provide euro and other symbols
\else % if luatex or xetex
  \usepackage{unicode-math}
  \defaultfontfeatures{Scale=MatchLowercase}
  \defaultfontfeatures[\rmfamily]{Ligatures=TeX,Scale=1}
\fi
% Use upquote if available, for straight quotes in verbatim environments
\IfFileExists{upquote.sty}{\usepackage{upquote}}{}
\IfFileExists{microtype.sty}{% use microtype if available
  \usepackage[]{microtype}
  \UseMicrotypeSet[protrusion]{basicmath} % disable protrusion for tt fonts
}{}
\makeatletter
\@ifundefined{KOMAClassName}{% if non-KOMA class
  \IfFileExists{parskip.sty}{%
    \usepackage{parskip}
  }{% else
    \setlength{\parindent}{0pt}
    \setlength{\parskip}{6pt plus 2pt minus 1pt}}
}{% if KOMA class
  \KOMAoptions{parskip=half}}
\makeatother
\usepackage{xcolor}
\usepackage[margin=1in]{geometry}
\usepackage{color}
\usepackage{fancyvrb}
\newcommand{\VerbBar}{|}
\newcommand{\VERB}{\Verb[commandchars=\\\{\}]}
\DefineVerbatimEnvironment{Highlighting}{Verbatim}{commandchars=\\\{\}}
% Add ',fontsize=\small' for more characters per line
\usepackage{framed}
\definecolor{shadecolor}{RGB}{248,248,248}
\newenvironment{Shaded}{\begin{snugshade}}{\end{snugshade}}
\newcommand{\AlertTok}[1]{\textcolor[rgb]{0.94,0.16,0.16}{#1}}
\newcommand{\AnnotationTok}[1]{\textcolor[rgb]{0.56,0.35,0.01}{\textbf{\textit{#1}}}}
\newcommand{\AttributeTok}[1]{\textcolor[rgb]{0.77,0.63,0.00}{#1}}
\newcommand{\BaseNTok}[1]{\textcolor[rgb]{0.00,0.00,0.81}{#1}}
\newcommand{\BuiltInTok}[1]{#1}
\newcommand{\CharTok}[1]{\textcolor[rgb]{0.31,0.60,0.02}{#1}}
\newcommand{\CommentTok}[1]{\textcolor[rgb]{0.56,0.35,0.01}{\textit{#1}}}
\newcommand{\CommentVarTok}[1]{\textcolor[rgb]{0.56,0.35,0.01}{\textbf{\textit{#1}}}}
\newcommand{\ConstantTok}[1]{\textcolor[rgb]{0.00,0.00,0.00}{#1}}
\newcommand{\ControlFlowTok}[1]{\textcolor[rgb]{0.13,0.29,0.53}{\textbf{#1}}}
\newcommand{\DataTypeTok}[1]{\textcolor[rgb]{0.13,0.29,0.53}{#1}}
\newcommand{\DecValTok}[1]{\textcolor[rgb]{0.00,0.00,0.81}{#1}}
\newcommand{\DocumentationTok}[1]{\textcolor[rgb]{0.56,0.35,0.01}{\textbf{\textit{#1}}}}
\newcommand{\ErrorTok}[1]{\textcolor[rgb]{0.64,0.00,0.00}{\textbf{#1}}}
\newcommand{\ExtensionTok}[1]{#1}
\newcommand{\FloatTok}[1]{\textcolor[rgb]{0.00,0.00,0.81}{#1}}
\newcommand{\FunctionTok}[1]{\textcolor[rgb]{0.00,0.00,0.00}{#1}}
\newcommand{\ImportTok}[1]{#1}
\newcommand{\InformationTok}[1]{\textcolor[rgb]{0.56,0.35,0.01}{\textbf{\textit{#1}}}}
\newcommand{\KeywordTok}[1]{\textcolor[rgb]{0.13,0.29,0.53}{\textbf{#1}}}
\newcommand{\NormalTok}[1]{#1}
\newcommand{\OperatorTok}[1]{\textcolor[rgb]{0.81,0.36,0.00}{\textbf{#1}}}
\newcommand{\OtherTok}[1]{\textcolor[rgb]{0.56,0.35,0.01}{#1}}
\newcommand{\PreprocessorTok}[1]{\textcolor[rgb]{0.56,0.35,0.01}{\textit{#1}}}
\newcommand{\RegionMarkerTok}[1]{#1}
\newcommand{\SpecialCharTok}[1]{\textcolor[rgb]{0.00,0.00,0.00}{#1}}
\newcommand{\SpecialStringTok}[1]{\textcolor[rgb]{0.31,0.60,0.02}{#1}}
\newcommand{\StringTok}[1]{\textcolor[rgb]{0.31,0.60,0.02}{#1}}
\newcommand{\VariableTok}[1]{\textcolor[rgb]{0.00,0.00,0.00}{#1}}
\newcommand{\VerbatimStringTok}[1]{\textcolor[rgb]{0.31,0.60,0.02}{#1}}
\newcommand{\WarningTok}[1]{\textcolor[rgb]{0.56,0.35,0.01}{\textbf{\textit{#1}}}}
\usepackage{graphicx}
\makeatletter
\def\maxwidth{\ifdim\Gin@nat@width>\linewidth\linewidth\else\Gin@nat@width\fi}
\def\maxheight{\ifdim\Gin@nat@height>\textheight\textheight\else\Gin@nat@height\fi}
\makeatother
% Scale images if necessary, so that they will not overflow the page
% margins by default, and it is still possible to overwrite the defaults
% using explicit options in \includegraphics[width, height, ...]{}
\setkeys{Gin}{width=\maxwidth,height=\maxheight,keepaspectratio}
% Set default figure placement to htbp
\makeatletter
\def\fps@figure{htbp}
\makeatother
\setlength{\emergencystretch}{3em} % prevent overfull lines
\providecommand{\tightlist}{%
  \setlength{\itemsep}{0pt}\setlength{\parskip}{0pt}}
\setcounter{secnumdepth}{5}
\usepackage{booktabs}
\usepackage{longtable}
\usepackage{array}
\usepackage{multirow}
\usepackage{wrapfig}
\usepackage{float}
\usepackage{colortbl}
\usepackage{pdflscape}
\usepackage{tabu}
\usepackage{threeparttable}
\usepackage{threeparttablex}
\usepackage[normalem]{ulem}
\usepackage{makecell}
\usepackage{xcolor}
\ifLuaTeX
  \usepackage{selnolig}  % disable illegal ligatures
\fi
\IfFileExists{bookmark.sty}{\usepackage{bookmark}}{\usepackage{hyperref}}
\IfFileExists{xurl.sty}{\usepackage{xurl}}{} % add URL line breaks if available
\urlstyle{same} % disable monospaced font for URLs
\hypersetup{
  pdftitle={Book exercises},
  hidelinks,
  pdfcreator={LaTeX via pandoc}}

\title{Book exercises}
\author{}
\date{\vspace{-2.5em}}

\begin{document}
\maketitle

{
\setcounter{tocdepth}{3}
\tableofcontents
}
\hypertarget{set-options}{%
\section*{Set options}\label{set-options}}
\addcontentsline{toc}{section}{Set options}

\hypertarget{chapter-1---introduction}{%
\section{Chapter 1 - Introduction}\label{chapter-1---introduction}}

\hypertarget{exercise-1.1-practice-using-the-pnorm-function---part-1}{%
\subsection{Exercise 1.1 Practice using the pnorm function - Part
1}\label{exercise-1.1-practice-using-the-pnorm-function---part-1}}

Given a normal distribution with mean 500 and standard deviation 100,
use the pnorm function to calculate the probability of obtaining values
between 200 and 800 from this distribution.

\begin{Shaded}
\begin{Highlighting}[]
\FunctionTok{pnorm}\NormalTok{(}\DecValTok{800}\NormalTok{, }\DecValTok{500}\NormalTok{, }\DecValTok{100}\NormalTok{) }\SpecialCharTok{{-}} \FunctionTok{pnorm}\NormalTok{(}\DecValTok{200}\NormalTok{, }\DecValTok{500}\NormalTok{, }\DecValTok{100}\NormalTok{)}
\end{Highlighting}
\end{Shaded}

\begin{verbatim}
## [1] 0.9973002
\end{verbatim}

\hypertarget{exercise-1.2-practice-using-the-pnorm-function---part-2}{%
\subsection{Exercise 1.2 Practice using the pnorm function - Part
2}\label{exercise-1.2-practice-using-the-pnorm-function---part-2}}

Calculate the following probabilities. Given a normal distribution with
mean 800 and standard deviation 150, what is the probability of
obtaining:

\begin{itemize}
\tightlist
\item
  a score of 700 or less
\item
  a score of 900 or more
\item
  a score of 800 or more
\end{itemize}

\begin{Shaded}
\begin{Highlighting}[]
\CommentTok{\# 700 or more}
\FunctionTok{pnorm}\NormalTok{(}\DecValTok{700}\NormalTok{,}\DecValTok{800}\NormalTok{,}\DecValTok{150}\NormalTok{)}
\end{Highlighting}
\end{Shaded}

\begin{verbatim}
## [1] 0.2524925
\end{verbatim}

\begin{Shaded}
\begin{Highlighting}[]
\CommentTok{\# 900, 800 or more}
\FunctionTok{pnorm}\NormalTok{(}\FunctionTok{c}\NormalTok{(}\DecValTok{900}\NormalTok{,}\DecValTok{800}\NormalTok{), }\DecValTok{800}\NormalTok{,}\DecValTok{150}\NormalTok{, }\AttributeTok{lower.tail=}\NormalTok{F)}
\end{Highlighting}
\end{Shaded}

\begin{verbatim}
## [1] 0.2524925 0.5000000
\end{verbatim}

\hypertarget{exercise-1.3-practice-using-the-pnorm-function---part-3}{%
\subsection{Exercise 1.3 Practice using the pnorm function - Part
3}\label{exercise-1.3-practice-using-the-pnorm-function---part-3}}

Given a normal distribution with mean 600 and standard deviation 200,
what is the probability of obtaining:

\begin{itemize}
\tightlist
\item
  a score of 550 or less.
\item
  a score between 300 and 800.
\item
  a score of 900 or more.
\end{itemize}

\begin{Shaded}
\begin{Highlighting}[]
\CommentTok{\# 550 or less}
\FunctionTok{pnorm}\NormalTok{(}\AttributeTok{q =} \DecValTok{550}\NormalTok{, }
      \AttributeTok{m =} \DecValTok{600}\NormalTok{, }\AttributeTok{sd =} \DecValTok{200}\NormalTok{)}
\end{Highlighting}
\end{Shaded}

\begin{verbatim}
## [1] 0.4012937
\end{verbatim}

\begin{Shaded}
\begin{Highlighting}[]
\CommentTok{\# 300{-}800}
\FunctionTok{pnorm}\NormalTok{(}\AttributeTok{q =} \DecValTok{800}\NormalTok{, }
      \AttributeTok{m =} \DecValTok{600}\NormalTok{, }\AttributeTok{sd =} \DecValTok{200}\NormalTok{) }\SpecialCharTok{{-}} \FunctionTok{pnorm}\NormalTok{(}\AttributeTok{q =} \DecValTok{300}\NormalTok{, }\AttributeTok{m =} \DecValTok{600}\NormalTok{, }\AttributeTok{sd =} \DecValTok{200}\NormalTok{)}
\end{Highlighting}
\end{Shaded}

\begin{verbatim}
## [1] 0.7745375
\end{verbatim}

\begin{Shaded}
\begin{Highlighting}[]
\CommentTok{\# 900 or more}
\FunctionTok{pnorm}\NormalTok{(}\AttributeTok{q =} \DecValTok{900}\NormalTok{, }
      \AttributeTok{m =} \DecValTok{600}\NormalTok{, }\AttributeTok{sd =} \DecValTok{200}\NormalTok{, }\AttributeTok{lower.tail=}\NormalTok{F)}
\end{Highlighting}
\end{Shaded}

\begin{verbatim}
## [1] 0.0668072
\end{verbatim}

\hypertarget{exercise-1.4-practice-using-the-qnorm-function---part-1}{%
\subsection{Exercise 1.4 Practice using the qnorm function - Part
1}\label{exercise-1.4-practice-using-the-qnorm-function---part-1}}

Consider a normal distribution with mean 1 and standard deviation 1.
Compute the lower and upper boundaries such that:

\begin{itemize}
\tightlist
\item
  the area (the probability) to the left of the lower boundary is 0.10.
\item
  the area (the probability) to the left of the upper boundary is 0.90.
\end{itemize}

\begin{Shaded}
\begin{Highlighting}[]
\CommentTok{\# lower bound is .1}
\FunctionTok{qnorm}\NormalTok{(}\AttributeTok{p =}\NormalTok{ .}\DecValTok{1}\NormalTok{,}
      \AttributeTok{mean =} \DecValTok{1}\NormalTok{, }\AttributeTok{sd =} \DecValTok{1}\NormalTok{)}
\end{Highlighting}
\end{Shaded}

\begin{verbatim}
## [1] -0.2815516
\end{verbatim}

\begin{Shaded}
\begin{Highlighting}[]
\CommentTok{\# lower bound is .9}
\FunctionTok{qnorm}\NormalTok{(}\AttributeTok{p =}\NormalTok{ .}\DecValTok{9}\NormalTok{,}
      \AttributeTok{mean =} \DecValTok{1}\NormalTok{, }\AttributeTok{sd =} \DecValTok{1}\NormalTok{)}
\end{Highlighting}
\end{Shaded}

\begin{verbatim}
## [1] 2.281552
\end{verbatim}

\hypertarget{exercise-1.5-practice-using-the-qnorm-function---part-2}{%
\subsection{Exercise 1.5 Practice using the qnorm function - Part
2}\label{exercise-1.5-practice-using-the-qnorm-function---part-2}}

Given a normal distribution with mean 650 and standard deviation 125.
There exist two quantiles, the lower quantile q1 and the upper quantile
q2, that are equidistant from the mean 650, such that the area under the
curve of the normal between q1 and q2 is 80\%. Find q1 and q2.

\begin{Shaded}
\begin{Highlighting}[]
\NormalTok{q1 }\OtherTok{\textless{}{-}} \FunctionTok{qnorm}\NormalTok{(}\AttributeTok{p =}\NormalTok{.}\DecValTok{1}\NormalTok{,}
            \AttributeTok{mean =} \DecValTok{650}\NormalTok{, }\AttributeTok{sd =} \DecValTok{125}\NormalTok{)}
\NormalTok{q2 }\OtherTok{\textless{}{-}} \FunctionTok{qnorm}\NormalTok{(}\AttributeTok{p =}\NormalTok{.}\DecValTok{9}\NormalTok{,}
            \AttributeTok{mean =} \DecValTok{650}\NormalTok{, }\AttributeTok{sd =} \DecValTok{125}\NormalTok{)}
\NormalTok{q1; q2}
\end{Highlighting}
\end{Shaded}

\begin{verbatim}
## [1] 489.8061
\end{verbatim}

\begin{verbatim}
## [1] 810.1939
\end{verbatim}

\begin{Shaded}
\begin{Highlighting}[]
\CommentTok{\# check this is right}
\FunctionTok{pnorm}\NormalTok{(}\AttributeTok{q =}\NormalTok{ q2, }
      \AttributeTok{m =} \DecValTok{650}\NormalTok{, }\AttributeTok{sd =} \DecValTok{125}\NormalTok{) }\SpecialCharTok{{-}} \FunctionTok{pnorm}\NormalTok{(}\AttributeTok{q =}\NormalTok{ q1, }\AttributeTok{m =} \DecValTok{650}\NormalTok{, }\AttributeTok{sd =} \DecValTok{125}\NormalTok{)}
\end{Highlighting}
\end{Shaded}

\begin{verbatim}
## [1] 0.8
\end{verbatim}

\hypertarget{exercise-1.6-practice-getting-summaries-from-samples---part-1}{%
\subsection{Exercise 1.6 Practice getting summaries from samples - Part
1}\label{exercise-1.6-practice-getting-summaries-from-samples---part-1}}

Given data that is generated as follows:

\begin{Shaded}
\begin{Highlighting}[]
\NormalTok{data\_gen1 }\OtherTok{\textless{}{-}} \FunctionTok{rnorm}\NormalTok{(}\DecValTok{1000}\NormalTok{, }\DecValTok{300}\NormalTok{, }\DecValTok{200}\NormalTok{)}
\end{Highlighting}
\end{Shaded}

Calculate the mean, variance, and the lower quantile q1 and the upper
quantile q2, that are equidistant and such that the range of probability
between them is 80\%.

\begin{Shaded}
\begin{Highlighting}[]
\NormalTok{mean }\OtherTok{\textless{}{-}} \FunctionTok{mean}\NormalTok{(data\_gen1)}
\NormalTok{sd }\OtherTok{\textless{}{-}} \FunctionTok{sd}\NormalTok{(data\_gen1)}

\NormalTok{q1 }\OtherTok{\textless{}{-}} \FunctionTok{qnorm}\NormalTok{(.}\DecValTok{1}\NormalTok{,}
\NormalTok{            mean,sd)}
\NormalTok{q2 }\OtherTok{\textless{}{-}} \FunctionTok{qnorm}\NormalTok{(.}\DecValTok{9}\NormalTok{,}
\NormalTok{            mean,sd)}
\NormalTok{q1; q2}
\end{Highlighting}
\end{Shaded}

\begin{verbatim}
## [1] 30.02346
\end{verbatim}

\begin{verbatim}
## [1] 551.8705
\end{verbatim}

\begin{Shaded}
\begin{Highlighting}[]
\CommentTok{\# check}
\FunctionTok{pnorm}\NormalTok{(q2,mean,sd) }\SpecialCharTok{{-}} \FunctionTok{pnorm}\NormalTok{(q1,mean,sd)}
\end{Highlighting}
\end{Shaded}

\begin{verbatim}
## [1] 0.8
\end{verbatim}

\hypertarget{exercise-1.7-practice-getting-summaries-from-samples---part-2.}{%
\subsection{Exercise 1.7 Practice getting summaries from samples - Part
2.}\label{exercise-1.7-practice-getting-summaries-from-samples---part-2.}}

This time we generate the data with a truncated normal distribution from
the package extraDistr. The details of this distribution will be
discussed later in section 4.1 and in the Box 4.1, but for now we can
treat it as an unknown generative process:

\begin{Shaded}
\begin{Highlighting}[]
\NormalTok{data\_gen1 }\OtherTok{\textless{}{-}}\NormalTok{ extraDistr}\SpecialCharTok{::}\FunctionTok{rtnorm}\NormalTok{(}\DecValTok{1000}\NormalTok{, }\DecValTok{300}\NormalTok{, }\DecValTok{200}\NormalTok{, }\AttributeTok{a =} \DecValTok{0}\NormalTok{)}
\end{Highlighting}
\end{Shaded}

Calculate the mean, variance, and the lower quantile q1 and the upper
quantile q2, that are equidistant and such that the range of probability
between them is 80\%.

\begin{Shaded}
\begin{Highlighting}[]
\NormalTok{mean }\OtherTok{\textless{}{-}} \FunctionTok{mean}\NormalTok{(data\_gen1)}
\NormalTok{sd }\OtherTok{\textless{}{-}} \FunctionTok{sd}\NormalTok{(data\_gen1)}

\NormalTok{q1 }\OtherTok{\textless{}{-}} \FunctionTok{qnorm}\NormalTok{(.}\DecValTok{1}\NormalTok{,}
\NormalTok{            mean,sd)}
\NormalTok{q2 }\OtherTok{\textless{}{-}} \FunctionTok{qnorm}\NormalTok{(.}\DecValTok{9}\NormalTok{,}
\NormalTok{            mean,sd)}
\NormalTok{q1; q2}
\end{Highlighting}
\end{Shaded}

\begin{verbatim}
## [1] 98.69408
\end{verbatim}

\begin{verbatim}
## [1] 573.8504
\end{verbatim}

\begin{Shaded}
\begin{Highlighting}[]
\CommentTok{\# check}
\FunctionTok{pnorm}\NormalTok{(q2,mean,sd) }\SpecialCharTok{{-}} \FunctionTok{pnorm}\NormalTok{(q1,mean,sd)}
\end{Highlighting}
\end{Shaded}

\begin{verbatim}
## [1] 0.8
\end{verbatim}

\hypertarget{exercise-1.8-practice-with-a-variance-covariance-matrix-for-a-bivariate-distribution.}{%
\subsection{Exercise 1.8 Practice with a variance-covariance matrix for
a bivariate
distribution.}\label{exercise-1.8-practice-with-a-variance-covariance-matrix-for-a-bivariate-distribution.}}

Suppose that you have a bivariate distribution where one of the two
random variables comes from a normal distribution with mean \(\mu\)X =
600 and standard deviation \(\sigma\)X = 100, and the other from a
normal distribution with mean \(\mu\)Y = 400 and standard deviation
\(\sigma\)Y = 50. The correlation \(\rho\)XY between the two random
variables is 0.4. Write down the variance-covariance matrix of this
bivariate distribution as a matrix (with numerical values, not
mathematical symbols), and then use it to generate 100 pairs of
simulated data points.

\begin{Shaded}
\begin{Highlighting}[]
\CommentTok{\# generate simulated bivariate data}

\DocumentationTok{\#\# define two RVs}

\DocumentationTok{\#\# define a VarCorr matrix, where rho = .6, variance}
\NormalTok{Sigma }\OtherTok{\textless{}{-}} \FunctionTok{matrix}\NormalTok{(}\FunctionTok{c}\NormalTok{(}
  \DecValTok{100}\SpecialCharTok{\^{}}\DecValTok{2}\NormalTok{, }\DecValTok{100} \SpecialCharTok{*} \DecValTok{50} \SpecialCharTok{*}\NormalTok{ .}\DecValTok{4}\NormalTok{, }
  \DecValTok{100} \SpecialCharTok{*} \DecValTok{50} \SpecialCharTok{*}\NormalTok{ .}\DecValTok{4}\NormalTok{, }\DecValTok{100}\SpecialCharTok{\^{}}\DecValTok{2} 
\NormalTok{  ),}
  \AttributeTok{byrow =}\NormalTok{ F, }\AttributeTok{ncol =} \DecValTok{2}
\NormalTok{  )}

\DocumentationTok{\#\# generate data}
\NormalTok{u }\OtherTok{\textless{}{-}} \FunctionTok{as.data.frame}\NormalTok{(MASS}\SpecialCharTok{::}\FunctionTok{mvrnorm}\NormalTok{(}\AttributeTok{n =} \DecValTok{100}\NormalTok{, }\AttributeTok{mu =} \FunctionTok{c}\NormalTok{(}\DecValTok{600}\NormalTok{,}\DecValTok{400}\NormalTok{), }\AttributeTok{Sigma =}\NormalTok{ Sigma))}
\FunctionTok{head}\NormalTok{(u, }\AttributeTok{n=}\DecValTok{3}\NormalTok{)}
\end{Highlighting}
\end{Shaded}

\begin{verbatim}
##         V1       V2
## 1 674.9871 426.9502
## 2 722.8999 295.1813
## 3 665.8666 395.5299
\end{verbatim}

Plot the simulated data such that the relationship between the random
variables X and Y is clear.

Generate two sets of new data (100 pairs of data points each) with
correlation −0.4 and 0, and plot these alongside the plot for the data
with correlation 0.4.

\begin{Shaded}
\begin{Highlighting}[]
\CommentTok{\# generate simulated bivariate data}
\NormalTok{rho }\OtherTok{\textless{}{-}} \SpecialCharTok{{-}}\NormalTok{.}\DecValTok{4}
\DocumentationTok{\#\# define a VarCorr matrix, where rho = 0, variance}
\NormalTok{Sigma }\OtherTok{\textless{}{-}} \FunctionTok{matrix}\NormalTok{(}\FunctionTok{c}\NormalTok{(}
  \DecValTok{100}\SpecialCharTok{\^{}}\DecValTok{2}\NormalTok{, }\DecValTok{100} \SpecialCharTok{*} \DecValTok{50} \SpecialCharTok{*}\NormalTok{ rho, }
  \DecValTok{100} \SpecialCharTok{*} \DecValTok{50} \SpecialCharTok{*}\NormalTok{ rho, }\DecValTok{100}\SpecialCharTok{\^{}}\DecValTok{2} 
\NormalTok{  ),}
  \AttributeTok{byrow =}\NormalTok{ F, }\AttributeTok{ncol =} \DecValTok{2}
\NormalTok{  )}

\DocumentationTok{\#\# generate data}
\NormalTok{u4 }\OtherTok{\textless{}{-}} \FunctionTok{as.data.frame}\NormalTok{(MASS}\SpecialCharTok{::}\FunctionTok{mvrnorm}\NormalTok{(}\AttributeTok{n =} \DecValTok{100}\NormalTok{, }\AttributeTok{mu =} \FunctionTok{c}\NormalTok{(}\DecValTok{600}\NormalTok{,}\DecValTok{400}\NormalTok{), }\AttributeTok{Sigma =}\NormalTok{ Sigma))}

\CommentTok{\# generate simulated bivariate data}
\NormalTok{rho }\OtherTok{\textless{}{-}} \DecValTok{0}
\DocumentationTok{\#\# define a VarCorr matrix, where rho = 0, variance}
\NormalTok{Sigma }\OtherTok{\textless{}{-}} \FunctionTok{matrix}\NormalTok{(}\FunctionTok{c}\NormalTok{(}
  \DecValTok{100}\SpecialCharTok{\^{}}\DecValTok{2}\NormalTok{, }\DecValTok{100} \SpecialCharTok{*} \DecValTok{50} \SpecialCharTok{*}\NormalTok{ rho, }
  \DecValTok{100} \SpecialCharTok{*} \DecValTok{50} \SpecialCharTok{*}\NormalTok{ rho, }\DecValTok{100}\SpecialCharTok{\^{}}\DecValTok{2} 
\NormalTok{  ),}
  \AttributeTok{byrow =}\NormalTok{ F, }\AttributeTok{ncol =} \DecValTok{2}
\NormalTok{  )}

\DocumentationTok{\#\# generate data}
\NormalTok{u0 }\OtherTok{\textless{}{-}} \FunctionTok{as.data.frame}\NormalTok{(MASS}\SpecialCharTok{::}\FunctionTok{mvrnorm}\NormalTok{(}\AttributeTok{n =} \DecValTok{100}\NormalTok{, }\AttributeTok{mu =} \FunctionTok{c}\NormalTok{(}\DecValTok{600}\NormalTok{,}\DecValTok{400}\NormalTok{), }\AttributeTok{Sigma =}\NormalTok{ Sigma))}
\end{Highlighting}
\end{Shaded}

\begin{Shaded}
\begin{Highlighting}[]
\NormalTok{ggpubr}\SpecialCharTok{::}\FunctionTok{ggarrange}\NormalTok{(}
\NormalTok{  ggplot2}\SpecialCharTok{::}\FunctionTok{ggplot}\NormalTok{(u, }\FunctionTok{aes}\NormalTok{(}\AttributeTok{x =}\NormalTok{ V1, }\AttributeTok{y =}\NormalTok{ V2)) }\SpecialCharTok{+}
    \FunctionTok{labs}\NormalTok{(}\AttributeTok{title =} \StringTok{"rho = .4"}\NormalTok{) }\SpecialCharTok{+} 
    \FunctionTok{geom\_point}\NormalTok{(),}
\NormalTok{  ggplot2}\SpecialCharTok{::}\FunctionTok{ggplot}\NormalTok{(u4, }\FunctionTok{aes}\NormalTok{(}\AttributeTok{x =}\NormalTok{ V1, }\AttributeTok{y =}\NormalTok{ V2)) }\SpecialCharTok{+}
    \FunctionTok{labs}\NormalTok{(}\AttributeTok{title =}\StringTok{"rho = .{-}4"}\NormalTok{) }\SpecialCharTok{+} 
    \FunctionTok{geom\_point}\NormalTok{(),}
\NormalTok{  ggplot2}\SpecialCharTok{::}\FunctionTok{ggplot}\NormalTok{(u0, }\FunctionTok{aes}\NormalTok{(}\AttributeTok{x =}\NormalTok{ V1, }\AttributeTok{y =}\NormalTok{ V2)) }\SpecialCharTok{+}
    \FunctionTok{labs}\NormalTok{(}\AttributeTok{title =}\StringTok{"rho = 0"}\NormalTok{) }\SpecialCharTok{+} 
    \FunctionTok{geom\_point}\NormalTok{(),}
  \AttributeTok{nrow =} \DecValTok{1}\NormalTok{, }\AttributeTok{labels =} \FunctionTok{c}\NormalTok{(}\StringTok{"A"}\NormalTok{, }\StringTok{"B"}\NormalTok{, }\StringTok{"C"}\NormalTok{)}
\NormalTok{)}
\end{Highlighting}
\end{Shaded}

\includegraphics{book_exercises_files/figure-latex/unnamed-chunk-13-1.pdf}

\end{document}
